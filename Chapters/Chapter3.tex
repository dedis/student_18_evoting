% Chapter Template

\chapter{Implementation} % Main chapter title

\label{Chapter3} % Change X to a consecutive number; for referencing this chapter elsewhere, use \ref{ChapterX}

\lhead{Chapter 3. \emph{Implementation}} % Change X to a consecutive number; this is for the header on each page - perhaps a shortened title

\section{Backend}

The backend for the evoting implementation makes extensive use of the Cothority framework and as such Go was the natural language of choice for implementing it. The project builds on Andrea Cafforio's Master Project work and is split across different packages.

\begin{itemize}
  \item \textbf{struct}: contains definitions of all messages used to interact with the backend using protocol buffers.
  \item \textbf{evoting-admin}: contains an administrative CLI application that is used to set up the Master Skipchain for the election.
  \item \textbf{lib}: contains all library methods and definitions for the evoting application including definitions for transaction structure, methods for storing data in the skipchain using the local skipchain service and over websockets, methods for performing ElGamal encryption, decryption, proving a neff shuffle and utility methods for different operations on the skipchain
  \item \textbf{protocol}: contains implementation of the DKG, Shuffle and Decrypt Protocols as described before.
  \item \textbf{service}: source for the evoting service which receives messages from the frontend client over protocol buffers and performs operations on the skipchain depending on the message type.
\end{itemize}

\section{Frontend}

The frontend client for this project is built using Vue.js \cite{vuejs}. Vue helps in rapid development of the user interface by allowing the composition of reusable components. It also allows efficient DOM updates by leveraging diffing in a Virtual DOM first and syncing only the required changes in the actual DOM.

Since the evoting sytem may be used on mobile devices as well, it was imperative to ensure the frontend is responsive for different screen sizes. In order to have a clean and modern UI, we leverage Vuetify \cite{vuetify}, a UI toolkit for Vue.js.

The routing between different pages of the application is handled using VueRouter \cite{vuerouter}, a routing module for Vue.js. We also use Vuex \cite{vuex} to manage states of different components, cache API responses in memory and in localStorage. The frontend is translated in English, French, German and Italian and uses vue-i18n \cite{vuei18n} for managing the same.

\subsection{KyberJS}
A critical part of voting process is to encrypt the ballots before storing them in the blockchain. While there is some support for cryptographic operations in browsers using javascript using the WebCrypto API \cite{webcrypto}, and in a Node.js environment using the Crypto API \cite{nodecrypto}, they are still in nascent stages and don't support Elliptic Curves we depend on (Ed25519) as of now. Since Javascript provides a unique opportunity to be used in browsers, server side and in native mobile applications (using frameworks like Nativescript \cite{nativescript} and ReactNative \cite{reactnative}), we built a wrapper around a library \cite{elliptic} providing Elliptic Curve Cryptography primitives like \emph{Point} and \emph{Scalar} operations in a finite prime field. The wrapper, called \textit{KyberJS} \cite{kyberjs} adopts an interface similar to its Go equivalent authored by the DEDIS Lab \cite{kyber}. This allows us to leverage a familiar API across browsers, servers and mobile targets and implement signature and encryption schemes with ease.

\subsection{CothorityJS}
As noted earlier, the set of conodes allow communication over Websockets using Protobufs to encode messages back and forth. In order to facilitate communication between any javascript client and conodes, we built a javascript library, called CothorityJS \cite{cothorityjs}. This library exposes protobuf communication to all the services supported on conodes, (with support for adding new protobuf messages with ease) using a Promise API familiar to javascript developers.